% ****** Start of file apssamp.tex ******
%
%   This file is part of the APS files in the REVTeX 4 distribution.
%   Version 4.0 of REVTeX, August 2001
%
%   Copyright (c) 2001 The American Physical Society.
%
%   See the REVTeX 4 README file for restrictions and more information.
%
% TeX'ing this file requires that you have AMS-LaTeX 2.0 installed
% as well as the rest of the prerequisites for REVTeX 4.0
%
% See the REVTeX 4 README file
% It also requires running BibTeX. The commands are as follows:
%
%  1)  latex apssamp.tex
%  2)  bibtex prb
%  3)  latex apssamp.tex
%  4)  latex apssamp.tex
%
%\documentclass[aps,prb,preprint,groupedaddress,showpacs]{revtex4-1}
\documentclass[aps,prl,preprint,superscriptaddress]{revtex4}
%\documentclass[aps,prl,twocolumn,superscriptaddress]{revtex4}
%\documentclass[aps,prl,twocolumn,superscriptaddress]{revtex4}
%\documentclass[aps,prb,twocolumn,groupedaddress]{revtex4-1}


%\documentclass[twocolumn,showpacs,preprintnumbers,amsmath,amssymb]{revtex4}
%\documentclass[preprint,showpacs,preprintnumbers,amsmath,amssymb]{revtex4}

% Some other (several out of many) possibilities
%\documentclass[preprint,aps]{revtex4}
%\documentclass[preprint,aps,draft]{revtex4}
%\documentclass[prb]{revtex4}% Physical Review B

\usepackage{graphics}
\usepackage{graphicx}% Include figure files
\usepackage{epstopdf}
\usepackage{dcolumn}% Align table columns on decimal point
\usepackage{bm}% bold math
\usepackage{amsmath}
\usepackage{amssymb}
\usepackage{latexsym}
\usepackage{epsfig}
\usepackage{amsbsy}
\usepackage{array}
\usepackage{amssymb}
\usepackage{setspace}
\usepackage{bm}
\usepackage{float}
\usepackage[caption = false]{subfig}

\newcommand{\ssint}{ - \!\!\!\!\! \int }
\def\sint{\ifmmode{- \!\!\!\!\!\! \int}
    \else{\hbox{$- \!\!\!\! \int \ $}}\fi}


\newcommand{\bsigma}{\boldsymbol{\sigma}}
\newcommand{\bmu}{\boldsymbol{\mu}}
\newcommand{\bvepsilon}{\boldsymbol{\varepsilon}}
\newcommand{\bepsilon}{\boldsymbol{\epsilon}}
\newcommand{\balpha}{\boldsymbol{\alpha}}
\newcommand{\bkappa}{\boldsymbol{\kappa}}
\newcommand{\bchi}{\boldsymbol{\chi}}
\newcommand{\bgamma}{\boldsymbol{\gamma}}
\newcommand{\bpsi}{\boldsymbol{\psi}}
\newcommand{\bnu}{\boldsymbol{\nu}}
\newcommand{\bzero}{\boldsymbol{0}}
\newcommand{\bbeta}{\boldsymbol{\beta}}
\newcommand{\bSigma}{\boldsymbol{\Sigma}}

\newcommand{\va}{\varphi}
\newcommand{\ep}{\epsilon}
\newcommand{\mbf}{{\bf m}}
\newcommand{\pbf}{{\bf p}}
\newcommand{\xbf}{{\bf x}}
\newcommand{\weak}{\rightharpoonup}
\newcommand{\rgoto}{\rightarrow}

\newcommand{\grad}{\mbox{grad}}
\newcommand{\curl}{\mbox{curl}}
\newcommand{\dive}{\mbox{div}}


\newcommand{\tr}{\mbox{tr}}

\newcommand{\ba}{\mathbf{a}}
\newcommand{\bb}{\mathbf{b}}
\newcommand{\bc}{\mathbf{c}}
\newcommand{\bd}{\mathbf{d}}
\newcommand{\be}{\mathbf{e}}
\newcommand{\bsf}{\mathbf{f}}
\newcommand{\bg}{\mathbf{g}}
\newcommand{\bsi}{\mathbf{i}}
\newcommand{\bk}{\mathbf{k}}
\newcommand{\bn}{\mathbf{n}}
\newcommand{\bo}{\mathbf{o}}
\newcommand{\bp}{\mathbf{p}}
\newcommand{\bq}{\mathbf{q}}
\newcommand{\br}{\mathbf{r}}
\newcommand{\bs}{\mathbf{s}}
\newcommand{\bt}{\mathbf{t}}
\newcommand{\bu}{\mathbf{u}}
\newcommand{\bv}{\mathbf{v}}
\newcommand{\bw}{\mathbf{w}}
\newcommand{\bx}{\mathbf{x}}
\newcommand{\by}{\mathbf{y}}
\newcommand{\bz}{\mathbf{z}}

\newcommand{\bca}{\mathbf{A}}
\newcommand{\bcb}{\mathbf{B}}
\newcommand{\bcc}{\mathbf{C}}
\newcommand{\bcd}{\mathbf{D}}
\newcommand{\bce}{\mathbf{E}}
\newcommand{\bcf}{\mathbf{F}}
\newcommand{\bcg}{\mathbf{G}}
\newcommand{\bch}{\mathbf{H}}
\newcommand{\bck}{\mathbf{K}}
\newcommand{\bcj}{\mathbf{J}}
\newcommand{\bci}{\mathbf{I}}
\newcommand{\bcl}{\mathbf{L}}
\newcommand{\bcm}{\mathbf{M}}
\newcommand{\bcn}{\mathbf{N}}
\newcommand{\bco}{\mathbf{O}}
\newcommand{\bcp}{\mathbf{P}}
\newcommand{\bcq}{\mathbf{Q}}
\newcommand{\bcr}{\mathbf{R}}
\newcommand{\bcs}{\mathbf{S}}
\newcommand{\bct}{\mathbf{T}}
\newcommand{\bcu}{\mathbf{U}}
\newcommand{\bcv}{\mathbf{V}}
\newcommand{\bcw}{\mathbf{W}}
\newcommand{\bcx}{\mathbf{X}}
\newcommand{\bcz}{\mathbf{Z}}
\newcommand{\bcy}{\mathbf{Y}}

%\nofiles

\begin{document}
	
	
	\title{Predator-Prey Model}% Force line breaks with \\
	
	\author{Connor Hann, Xiaomeng Jia, Peifan Liu and Xinyu Wu}
	\affiliation{Physics Department, Duke University}
	
	
	\date{\today}
	
	\begin{abstract}
		To be added.
	\end{abstract}
	
	\maketitle
	
	
	
\section{Background} 
Modeling the interactions between predator and prey is a question of great ecological importance, as accurate models can allow one to make reliable predictions and thereby inform decisions in conservation, habitat preservation, hunting regulations, etc. There exist a variety of methods for modeling the interactions between predator and prey, and these models, though simple, are often in quite good agreement with what is observed. Take the classic Lotka-Volterra Model as an example. While this model is nothing more than a set of first-order nonlinear differential equations that one may solve numerically for a particular set of initial conditions, the model is capable of providing a qualitatively accurate description of the data, as is shown in Fig.~\ref{LV}. 

\begin{figure}[H]
	\centering
	\subfloat{\includegraphics[width = 0.7\textwidth]{LV_data}}\\
	\subfloat{\includegraphics[width = 0.7\textwidth]{LV_prediction}}
	\caption{(Top) experimental data of Hare and Lynx populations, (Bottom) predictions from the Lotka-Volterra equations. }
	\label{LV} 
\end{figure}

In this work we study the behavior of a slightly different model, one that  is similar to the Lotka-Volterra Model in that it consists of only a simple set of rules, but the nature of the actual simulation is quite different. In our case, we consider two populations, sharks and fish living on a two-dimensional lattice. We simulate the movement and behavior of each shark and fish individually and can observe fluctuations of the two populations as a function of time. Unlike the Lotka-Volterra Model, however, we are able to see exactly where all of the sharks and fish are and this allows us to attain a deeper understanding of what causes the fluctuations in populations and the conditions that may lead to extinction. 


\section{Classical Lotka-Volterra Model}	
The classical Lotka-Volterra Model is given by:

\begin{equation}
x' = x(\alpha-\beta y),
\end{equation}
\begin{equation}
y' = -y(\gamma -\delta x),
\end{equation}
where $\alpha x$ is the growth term which leads to exponential growth of prey population in absence of predators; $-\beta yx$ is the loss term for prey which depends both on numbers of predators $y$ and number of prey $x$; $-\gamma y$ is the loss term for predators in the absence of prey; $\delta yx$ is the gain term for predators which depends both on numbers of predators $y$ and number of prey $x$. 

\pagebreak
\section{Implementation}

The model we consider is remarkably simple in that there are only five user input parameters: the initial number of sharks $n0\_sharks$, initial number of fish $n0\_fish$, the time it takes fish to breed $breed\_age\_fish$, the time it takes sharks to breed $breed\_age\_shark$, and the time it takes a shark to starve $starve\_time$. To begin the simulation, one creates $n0\_sharks$ sharks and $n0\_fish$ fish at random locations on a two-dimensional lattice, and each fish or shark is given a random age between $breed\_age\_fish$ and $breed\_age\_shark$, respectively. At each subsequent time step of the simulation both sharks and fish will swim around the grid, and the sharks will attempt to eat the fish. Below we discuss the specific rules that dictate the movements of the fish and sharks.

\subsection{Fish Movement}
During a time step, each fish looks at the four adjacent lattice sites and chooses an empty site to which to move. If no lattice sites are available, the fish remains in its current position. We implement periodic boundary conditions so that if a fish moves through one boundary it reappears on the boundary on the opposite side, i.e. our two-dimensional grid is actually a topological torus. The age of the fish is also increased by one. If the age of the fish is $breed\_age\_Fish$, an additional fish is created at the site the fish has just moved from, and the ages of both fish are set to 0.

\subsection{Shark Movement}
The goal of the sharks is to eat fish, so during a time step each shark will look for fish on the adjacent lattice sites. If fish are found, the shark chooses one randomly and moves to the fish's position on the lattice, deleting and ``eating'' the fish. If no fish are available the sharks move just like the fish, choosing an available lattice site to move to randomly. Like the fish, the age of the sharks is increased by one each time step, and sharks breed in the same way as the fish once they have reached $breed\_age\_Shark$. Unlike the fish, however, the sharks will starve if they do not find food. An array keeps track of how long it has been since each shark has last eaten a fish, and if a shark has not found a fish within $starve\_time$ time steps, it ``starves'' and is deleted from the simulation.

\subsection{Program Structure}
We use a set of five arrays to keep track of the information necessary to run the simulation as described above. Two $N \times N$ arrays, $Sharks$ and $Fish$, are used to keep track of the positions of the sharks and fish respectively. If the $(i,j)$ site is empty the arrays hold the value $-1$, otherwise the site holds the age of the shark or fish. Two additional arrays $Sharkmove$ and $Fishmove$ are used to store the new positions of the sharks and fish as an update is being performed, allowing us to ensure that each shark or fish is moved only once and that two animals are not moved onto the same lattice site. Finally, a fifth array $Sharkstarve$ stores the amount of time since each shark has last eaten. The arrays are updated, first fish then sharks, via the rules above. One can plot the positions of the sharks and fish at any time step, and an example plot is shown in Fig.~\ref{ex}.

\begin{figure}[H]
	\centering
	\includegraphics[width = 0.7\textwidth]{example}
	\caption{A typical configuration of sharks and fish on a $20 \times 20$ lattice}
	\label{ex} 
\end{figure}

\pagebreak
\section{Numerical Results}

By choosing a $100*100$ 2D torus grid where 2000 fish and 200 sharks of random ages are randomly put as initial condition, it can be seen that as time goes on, the populations of fish and shark experience periodic variations. The number of fish increases most significantly when number of sharks reaches a valley, and decreases most sharply when shark reaches its peaks, as shown in Fig.3. The phase plane is shown in Fig.4. If we trace the dynamics of the system in each time step, it can be seen that sometimes fish fill out the whole grid and sometimes become almost extinct(Fig.6-7). This reminds us that the capacity of the grid is the upper constraint for the fish number, so the classical Lotka-Volterra equations must be modified to adapt to this simulation. Instead of exponential growth, the fish number should be described by logistic equation which adds the environment capacity K, and so does the sharks:

\begin{equation}
x' = x(A - Bx - Cy)
\end{equation}
\begin{equation}
y' = -y(D - EY -Fx)
\end{equation}
which gives phase diagram (Fig.5) that properly coincide with simulation result (Fig.4).

\begin{figure}[H]
	\centering
	\subfloat{\includegraphics[width = 0.7\textwidth]{pop3.jpg}}
	\caption{Population evolution of fish and sharks.}
	\label{more_clusters} 
\end{figure}


\begin{figure}[H]
	\centering
	\subfloat{\includegraphics[width = 0.8\textwidth]{phase_portrait.pdf}}	
	\caption{Population evolution of fish and sharks.}
	\label{more_clusters} 
\end{figure}

\begin{figure}[H]
	\centering
	\subfloat{\includegraphics[width = 0.7\textwidth]{PreyPredator.pdf}}
	\caption{Phase diagram of modified Lotka-Volterra equations($Here: A=C=F=1,\:B=0.3, D=0.8,E=0.2$).}
	\label{more_clusters} 
\end{figure}

\begin{figure}[H]
	\centering
	\subfloat{\includegraphics[width = 0.33\textwidth]{1.jpg}}
	\subfloat{\includegraphics[width = 0.33\textwidth]{46.jpg}}
	\subfloat{\includegraphics[width = 0.33\textwidth]{98.jpg}}	
	\caption{Population evolution of fish and sharks.}
	\label{more_clusters} 
\end{figure}

\begin{figure}[H]
	\centering
	\subfloat{\includegraphics[width = 0.33\textwidth]{222.jpg}}
	\subfloat{\includegraphics[width = 0.33\textwidth]{324.jpg}}
	\subfloat{\includegraphics[width = 0.33\textwidth]{463.jpg}}
	\caption{Population evolution of fish and sharks.}
	\label{more_clusters} 
\end{figure}

We can see from Fig. 5 that the system has an unstable fixed point, if the initial conditions are far away from the stable point, extinction of sharks or fish can be observed (see Fig.8-9). Fig. 8 shows the extinction of both shark and fish, we can see from this figure that after the extinction of fish, the shark goes through a part of the logistic decrease. Fig. 9 shows the extinction of shark, and we can see that fish goes through a logistic increase given the limited total grid points.


\begin{figure}[H]
	\centering
	\subfloat{\includegraphics[width = 0.7\textwidth]{pop5.pdf}}
	\caption{Extinction of both fish and sharks.}
	\label{more_clusters} 
\end{figure}

\begin{figure}[H]
	\centering
	\subfloat{\includegraphics[width = 0.7\textwidth]{pop6.pdf}}
	\caption{Extinction of sharks.}
	\label{more_clusters} 
\end{figure}
 
\section{Conclusion}
To conclude, we simulated the prey-predator model based on certain rules for the movements of both species, and the results we got coincides with the theoretical   expectations.
\end{document}
	
	%
	% ****** End of file apssamp.tex ******
